\title{Chapter 4, Section 3. Exercise 1 only.}
\author{
	MTH 594, Prof. Mikael Vejdemo-Johansson \\
	Differential Geometry Independent Study \\
	\\
	Matthew Connelly \\
}
\date{\today}



\documentclass[12pt]{article}

\usepackage[top=.5in, bottom=.75in, left=1in, right=1in]{geometry}
\usepackage{amssymb}
\usepackage{amsmath}
\usepackage{graphicx}
\usepackage{subcaption}

\newcommand{\ulind}[1]
{
\noindent
\underline{#1}\\\\
\indent
}

\newcommand{\R}
{
\mathbb{R}
}

\begin{document}
\maketitle

\section*{Exercise 4.3.1}
\indent

If $S$ is a smooth surface, define the notion of a \emph{smooth function} $S\rightarrow\R$. Show that, if $S$ is a smooth surface, each component of the inclusion map $S\rightarrow\R^3$ is a smooth function $S\rightarrow\R$.

\vspace{1cm}
\hrule
\vspace{1cm}
\noindent

\ulind{Smooth Function $S\rightarrow\R$}
Surface $S$ can be parametrized as follows:
$$
\sigma : U \rightarrow \R^3
$$
Then, to define a smooth function $f$ that will map $S$ to $\R$ we must be able to map $U$ to $R$ smoothly, which can be done with the following composition:
$$
f \circ \sigma : U \rightarrow \R
$$
or more explicitly:
$$
f(\sigma(U)) = \R
$$
$$
f(\R^3) = \R.
$$

\ulind{Inclusion Map $S\rightarrow\R^3$}
Let some function $g$ be defined as:
$$
g: S \rightarrow \R^3
$$
\indent
If $g$ is an inclusion map, then $\forall s \in S$, $g(s) = s$. That is, $g(s) \subset \R^3$, as is to be expected of a $\R^3$ surface. \\
\indent
If $g(s)$ were surjective (meaning, if $g(s) \subseteq \R^3$), then it would be an identity function $I(s \in \R^3) = s$.\\
\indent
$g(s)$ can then be written as a vector-valued function: $g(s) = <x, y, z>$, \\where $s_n = <x_n,y_n,z_n>$ (as implied above with the mention of $g(s) = s$). $\therefore$, each component of $s$, being scalar and belonging to $\R$, will then be mapped to $\R \in S$.

\end{document}
