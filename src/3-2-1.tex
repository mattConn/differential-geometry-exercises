\title{Chapter 3, Section 2. Exercises 1 and 2}
\author{
	MTH 594, Prof. Mikael Vejdemo-Johansson \\
	Differential Geometry Independent Study \\
	\\
	Matthew Connelly \\
}
\date{\today}



\documentclass[12pt]{article}

\usepackage[top=.5in, bottom=.75in, left=1in, right=1in]{geometry}
\usepackage{amssymb}
\usepackage{amsmath}
\usepackage{graphicx}
\usepackage{subcaption}


\begin{document}
\maketitle

\section*{Exercise 3.2.1}

Show that the length $l(\gamma)$ and the area $A(\gamma)$ are unchanged by applying an isometry to $\gamma$.

\vspace{1cm}
\hrule
\vspace{1cm}

By definition, $l(\gamma)$ cannot be changed by an isometry, for an isometric transformation must preserve length.\\\\
\indent
Following this rule, the isoperimetric inequality will remain intact, based on its dependence on $l(\gamma)$:

$$
A(\gamma) \ \leq \ \frac{l(\gamma)^2}{4\pi}
$$
Where length will be equal to:
$$
l(\gamma) \ = \ \int_{0}^{T} ||\gamma'|| \ dt
$$
$T$ being the period of the curve.
\end{document}
This is never printed