\title{Chapter 4, Section 4. Exercises 1, 2, and 4.}
\author{
	MTH 594, Prof. Mikael Vejdemo-Johansson \\
	Differential Geometry Independent Study \\
	\\
	Matthew Connelly \\
}
\date{\today}



\documentclass[12pt]{article}

\usepackage[top=.5in, bottom=.75in, left=1in, right=1in]{geometry}
\usepackage{amssymb}
\usepackage{amsmath}
\usepackage{graphicx}
\usepackage{subcaption}

\newcommand{\ulind}[1]
{
\noindent
\underline{#1}\\\\
\indent
}

\newcommand{\R}
{
\mathbb{R}
}

\begin{document}
\maketitle

\section*{Exercise 4.4.4}
\indent
Let $f: S_1 \rightarrow S_2$ be a local diffeomorphism and let $\gamma$ be a regular curve on $S_1$. Show that $f \circ \gamma$ is a regular curve on $S_2$.

\vspace{1cm}
\hrule
\vspace{1cm}
\noindent

$f$ maps a surface $S_1 \in \R^3$ to another surface $S_2 \in \R^3$. $f \circ \gamma$ can also be written as

$$
f( \gamma ) = S_2.
$$
\indent
More explicitly, a local diffeomorphism will map all $p \in S_1$ to some $\tilde{p} \in S_2$.\\
\indent
Knowing that all $p \in \gamma$ will also be in $S_1$, $\gamma$ will always be mapped to $\tilde{\gamma} \in S_2$ inadvertently via $f$, so long as it is regular in $S_1$ (as in, never breaking).

\end{document}
