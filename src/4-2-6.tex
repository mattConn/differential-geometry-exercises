\title{Chapter 4, Section 2. Exercises 1, 2, and 4 through 9}
\author{
	MTH 594, Prof. Mikael Vejdemo-Johansson \\
	Differential Geometry Independent Study \\
	\\
	Matthew Connelly \\
}
\date{\today}



\documentclass[12pt]{article}

\usepackage[top=.5in, bottom=.75in, left=1in, right=1in]{geometry}
\usepackage{amssymb}
\usepackage{amsmath}
\usepackage{graphicx}
\usepackage{subcaption}

\newcommand{\ulind}[1]
{
\noindent
\underline{#1}\\\\
\indent
}


\begin{document}
\maketitle

\section*{Exercise 4.2.6}
\indent
A \emph{helicoid} is the surface swept out by an aeroplane propeller, when both the aeroplane and its propeller move at constant speed. If the aeroplane is flying along the $z$-axis, show that the helicoid can be parametrized as
$$
\sigma(u, v) = (v \cos u, v \sin u, \lambda u)
$$
where $\lambda$ is a constant. Show that the cotangent of the angle that the standard unit normal of $\sigma$ at a point $p$ makes with the $z$-axis is proportional to the distance of $p$ from the $z$-axis.

\vspace{1cm}
\hrule
\vspace{1cm}
\noindent

\ulind{Finding $\sigma$'s unit normal}
$\sigma$'s unit normal can be computed as follows:

$$
\hat{n} = \frac{\sigma_u \times \sigma_v}{||\sigma_u \times \sigma_v||}
$$

But first, its normal is:

$$
n = (-\lambda \sin u, \lambda \cos u, -v)
$$

And the norm of the normal is:

$$
||n|| = \sqrt{\lambda^2 \sin^2 u + \lambda^2 \cos^2 u + v^2}\\
$$
$$
= \sqrt{\lambda^2 + v^2}
$$

$\therefore$, the unit normal to $\sigma$ is:
$$
\hat{n} = \left(\frac{-\lambda \sin u}{\sqrt{\lambda^2+v^2}}, \frac{\lambda \cos u}{\sqrt{\lambda^2+v^2}}, \frac{-v}{\sqrt{\lambda^2+v^2}}\right)
$$

\end{document}
