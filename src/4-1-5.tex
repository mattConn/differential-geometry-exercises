\title{Chapter 4, Section 1. Exercises 1, 2, 3 and 5}
\author{
	MTH 594, Prof. Mikael Vejdemo-Johansson \\
	Differential Geometry Independent Study \\
	\\
	Matthew Connelly \\
}
\date{\today}



\documentclass[12pt]{article}

\usepackage[top=.5in, bottom=.75in, left=1in, right=1in]{geometry}
\usepackage{amssymb}
\usepackage{amsmath}
\usepackage{graphicx}
\usepackage{subcaption}


\begin{document}
\maketitle

\section*{Exercise 4.1.5}
Show that every open subset of a surface is a surface.

\vspace{1cm}
\hrule
\vspace{1cm}

Let $S \in \mathbb{R}^3$ be a surface.\\
Let $U \in \mathbb{R}^2$ and $W \in \mathbb{R}^3$ be open sets in their respective spaces.\\
\\
\indent
Then, $S \cap W$ will be a surface patch (and an open subset of surface $S$) if $U$ can be mapped to $S \cap W$ smoothly and bijectively.\\
\indent
$\sigma$ will be this mapping function:
$$
\sigma: U \rightarrow S \cap W
$$

Then, let there be a group of subsets, defined as
$$ S_2 \subseteq \left( S \cap W \right)$$
$$ W_2 \subseteq W$$
$$ U_2 \subseteq U$$
and, a smooth, bijective mapping function analogous to $\sigma$:
$$
\sigma_2: U_2 \rightarrow S_2
$$

The following comparison can now be made:

$$
S_2 \cap W_2 = \left( S \cap W \right) \cap W_2
$$

Which means that the open subset of surface $S$ is a surface because it can be mapped to a surface patch homeomorphically. This general definition can then be applied to all surface subsets.

\end{document}
This is never printed