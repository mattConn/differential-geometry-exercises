\title{Chapter 4, Section 2. Exercises 1, 2, and 4 through 9}
\author{
	MTH 594, Prof. Mikael Vejdemo-Johansson \\
	Differential Geometry Independent Study \\
	\\
	Matthew Connelly \\
}
\date{\today}



\documentclass[12pt]{article}

\usepackage[top=.5in, bottom=.75in, left=1in, right=1in]{geometry}
\usepackage{amssymb}
\usepackage{amsmath}
\usepackage{graphicx}
\usepackage{subcaption}

\newcommand{\ulind}[1]
{
\noindent
\underline{#1}\\\\
\indent
}

\newcommand{\norm}[1]
{
||#1||
}


\begin{document}
\maketitle

\section*{Exercise 4.2.7}
\indent

Let $\gamma$ be a unit-speed curve in $\mathbb{R}^3$ with nowhere vanishing curvature. The \emph{tube} of radius $a > 0$ around $\gamma$ is the surface parametrized by
$$
\sigma(s,\theta) = \gamma(s) + a( \ n(s)\cos \theta + b(s) \sin \theta \ ),
$$

where $n$ is the principal normal of $\gamma$ and $b$ is its binormal. Give a geometrical description of this surface. Prove that $\sigma$ is regular if the curvature $k$ of $\gamma$ is less than $a^{-1}$ everywhere.\\

Note that, even if $\sigma$ is regular, the surface $\sigma$ will have self-intersections if the curve $\gamma$ comes within a distance $2a$ of itself. This illustrates the fact that regularity is a \emph{local} property: if $(s,\theta)$ is restricted to lie in a sufficiently small open subset of $U$ of $\mathbb{R}^2$, $\sigma : U \rightarrow \mathbb{R}^3$ will be smooth and injective (so there will be no self-intersections) - see Exercise 5.6.3. We shall see other instances of this later (for example, Example 12.2.5).

\vspace{1cm}
\hrule
\vspace{1cm}
\noindent

\ulind{Preliminaries}
Given that $\gamma$ is defined as unit-speed, we know that $\norm{\gamma'} = 1$, which then gives way to the following definitions:

$$
n(s) = \frac{T'}{\norm{T'}}
$$

where

$$
T = \gamma'.
$$

A more explicit definition can then be given:

$$
n(s) = \frac{\gamma''}{\norm{\gamma''}}
$$

\clearpage

Lastly, our binormal will read as follows:

$$
b(s) = T \times N
$$
$$
= \gamma' \times \frac{\gamma''}{\norm{\gamma''}} = 1
$$

Because $T$ and $N$ are both unit vectors, their cross product will be $1$.\\

\ulind{A Geometric Description of $\sigma$}
$\sigma$ can be described as a tube swept out by a circle of radius $a$ in a plane perpendicular to $\gamma$ as its center travels along $\gamma$. \\

\ulind{Proving $\sigma$ is Regular}
We are tasked with proving that $\sigma$ is regular, so long as the curvature $k$ of $\gamma$ is less than the inverse of $\sigma$'s radius everywhere.


\end{document}
