\title{Chapter 4, Section 1. Exercises 1, 2, 3 and 5}
\author{
	MTH 594, Prof. Mikael Vejdemo-Johansson \\
	Differential Geometry Independent Study \\
	\\
	Matthew Connelly \\
}
\date{\today}



\documentclass[12pt]{article}

\usepackage[top=.5in, bottom=.75in, left=1in, right=1in]{geometry}
\usepackage{amssymb}
\usepackage{amsmath}
\usepackage{graphicx}
\usepackage{subcaption}


\begin{document}
\maketitle

\section*{Exercise 4.1.1}

Show that any open disc in the $xy$-plane is a surface.

\vspace{1cm}
\hrule
\vspace{1cm}

Let an open disc in the $xy$-plane be defined as
$$
D = \lbrace (x,y) \in \mathbb{R}^2  \ \vert  \ \Vert (x,y)-a \Vert < r \rbrace
$$
where $a$ is the disc's center and $r$ is its radius $> 0$.\\\\
Then, let there be an open ball in $\mathbb{R}^3$ defined similarly as 
$$
W = \lbrace (x,y,z) \in \mathbb{R}^3  \ \vert  \ \Vert (x,y,z)-a \Vert < r \rbrace
$$
with $a$ and $r$ representing its own center and radius.\\
\indent
By definition, $D \subset W$, which implies $D \subset \mathbb{R}^3$; this will eliminate the need to define a separate open disc in $\mathbb{R}^2$ for the following statement on homeomorphism.\\
\\
We can then see that
$$
D \cap W = D
$$	
which means that $D \cap W$ is homeomorphic to $D$.\\\\
Or, in more formal notation:
$$
\sigma : D \rightarrow D \cap W
$$

and

$$
\sigma^{-1} : D \cap W \rightarrow D
$$

This second statement inverting $\sigma$ is especially important; mapping one set to its intersection with another set is trivial, but the inversion of that mapping may not yield a continuous result with preservation of range. Here, however, we are successful in mapping a surface to an open disc and vice versa.\\
\indent
It should also be noted that the atlas of $D$ is a single surface patch, reinforcing the notion that the intersection of $D$ and some $\mathbb{R}^3$ set will be homeomorphic to itself ($D$) for all points in $\mathbb{R}^2$ that are also in $D$.
\end{document}
This is never printed