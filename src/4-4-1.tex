\title{Chapter 4, Section 4. Exercises 1, 2, and 4.}
\author{
	MTH 594, Prof. Mikael Vejdemo-Johansson \\
	Differential Geometry Independent Study \\
	\\
	Matthew Connelly \\
}
\date{\today}



\documentclass[12pt]{article}

\usepackage[top=.5in, bottom=.75in, left=1in, right=1in]{geometry}
\usepackage{amssymb}
\usepackage{amsmath}
\usepackage{graphicx}
\usepackage{subcaption}

\newcommand{\ulind}[1]
{
\noindent
\underline{#1}\\\\
\indent
}

\newcommand{\R}
{
\mathbb{R}
}

\begin{document}
\maketitle

\section*{Exercise 4.4.1}
\indent

Find the equation of the tangent plane of each of the following surface patches at the indicated points:\\

$$
(i) \ \sigma(u,v) = (u,v,u^2-v^2), \ (1,1,0)
$$
$$
(ii) \ \sigma(r,\theta) = (r \cosh \theta, r \sinh \theta,r^2), \ (1,0,1)
$$

\vspace{1cm}
\hrule
\vspace{1cm}
\noindent

\ulind{Preliminaries}

The equation of a tangent plane to a surface $z = f(x,y)$ at point $(x_0,y_0,z_0)$ is as follows:
$$
z = f_x(x_0,y_0)(x-x_0)+f_y(x_0,y_0)(y-y_0)+z_0
$$

\ulind{Solutions}
(i):\\
The partial derivatives of $\sigma$ are
$$
\sigma_u = (1,0,2u), \ \sigma_v = (0,1,-2v). 
$$
Computing $\sigma_u$ and $\sigma_v$ at $(1,1,0)$ gives
$$
\sigma_u(1,1,0) = (1,0,2), \ \ \sigma_v(1,1,0) = (0,1,-2). 
$$
Crossing these partials gives us our normal:
$$
\sigma_u \times \sigma_v = (-2,2,1)
$$
Our tangent plane will then be
$$
-2(x-1)+2(y-1)+(z)=0
$$

$$
(-2x+2)+(2y-2)+(z)=0
$$

$$
-2x+2y+z=0
$$

\end{document}
