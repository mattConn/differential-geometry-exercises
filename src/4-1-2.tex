\title{Chapter 4, Section 1. Exercises 1, 2, 3 and 5}
\author{
	MTH 594, Prof. Mikael Vejdemo-Johansson \\
	Differential Geometry Independent Study \\
	\\
	Matthew Connelly \\
}
\date{\today}



\documentclass[12pt]{article}

\usepackage[top=.5in, bottom=.75in, left=1in, right=1in]{geometry}
\usepackage{amssymb}
\usepackage{amsmath}
\usepackage{graphicx}
\usepackage{subcaption}


\begin{document}
\maketitle

\section*{Exercise 4.1.2}

Define surface patches $\sigma^{x}_{\pm}:U \rightarrow \mathbb{R}^3$ for $S^2$ by solving the equation $x^2 + y^2 + z^2 = 1$ for $x$ in terms of  $y$ and $z$:
$$
\sigma^{x}_{\pm}(u,v) = (\pm \sqrt{1-u^2-v^2}, u ,v)
$$
\\
defined on the open set $U = \lbrace (u,v) \in \mathbb{R}^2 \ \vert \ u^2 + v^2 < 1\rbrace$. Define $\sigma^{y}_{\pm}$ and $\sigma^{z}_{\pm}$ similarly (with the same $U$) by solving for $y$ and $z$, respectively. Show that these six patches give $S^2$ the structure of a surface.
\vspace{1cm}
\hrule
\vspace{1cm}

\underline{Surface patch definition for $S^2$:}\\

The six surface patches for $S^2$ are
$$
x = \pm\sqrt{1-z^2-y^2}
$$
$$
y = \pm\sqrt{1-z^2-x^2}
$$
$$
z = \pm\sqrt{1-y^2-x^2}
$$
which can be parametrized, as a stated in the problem, as
$$
\sigma^{x}_{\pm}(u,v) = (\pm \sqrt{1-u^2-v^2}, u ,v)
$$
$$
\sigma^{y}_{\pm}(u,v) = (u, \pm \sqrt{1-u^2-v^2} ,v)
$$
$$
\sigma^{z}_{\pm}(u,v) = (u ,v, \pm \sqrt{1-u^2-v^2})
$$
\\
\indent
Then, for the open set $U$ defined in the problem as $u^2 + v^2 < 1$, which also looks like $Int(x^2+y^2 = 1)$,

$$
\sigma : U \rightarrow S^2.
$$
\\
\underline{Proof that $S^2$ is a surface:}\\
\indent
For any of the six patches , say, $x = +\sqrt{1-z^2-y^2}$, there will be a point $p \in U$ that can be mapped to $S^2$ by $\sigma^{x}_{+}$. The union of patches $\sigma^{x}_{+}$ and $\sigma^{x}_{-}$, will then make a surface, as will the union of any other pair of homeomorphisms of opposing signs from the six patches.

\end{document}
This is never printed