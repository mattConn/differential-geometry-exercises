\title{Chapter 4, Section 2. Exercises 1, 2, and 4 through 9}
\author{
	MTH 594, Prof. Mikael Vejdemo-Johansson \\
	Differential Geometry Independent Study \\
	\\
	Matthew Connelly \\
}
\date{\today}



\documentclass[12pt]{article}

\usepackage[top=.5in, bottom=.75in, left=1in, right=1in]{geometry}
\usepackage{amssymb}
\usepackage{amsmath}
\usepackage{graphicx}
\usepackage{subcaption}

\newcommand{\ulind}[1]
{
\noindent
\underline{#1}\\\\
\indent
}


\begin{document}
\maketitle

\section*{Exercise 4.2.5}
\indent
A \emph{torus} is obtained by rotating a circle $C$ in a plane $\Pi$ around a straight line $l$ in $\Pi$ that does not intersect $C$. Take $\Pi$ to be the $xz$-plane, $l$ to be the $z$-axis, $a>0$ the distance of the centre of $C$ from $l$, and $b < a$ the radius of $C$. Show that the torus is a smooth surface with parametrization
$$
\sigma(\theta,\phi) = ( \ (a+b\cos \theta)\cos \phi, (a+b\cos \theta)\sin \phi, b \sin \theta \ )
$$

\vspace{1cm}
\hrule
\vspace{1cm}
\noindent

\ulind{Initial observations}

Looking at the intersection of $\sigma$ with the $xz$-plane when $\phi \approx 0$ (meaning, the $xz$-plane would have not quite begun rotating yet), will give us the following image:

\begin{figure}[h!]
\centering
	\begin{subfigure}[b]{0.5\linewidth}
		\includegraphics[width=\linewidth]{./assets/4-2-5/torus-xz-intersection.png}
	\end{subfigure}
\end{figure}
\indent


Pictured is circle $C$ (not labeled), non-zero distance $a$ between axis of rotation $z$ and the center of $C$, and radius $b$, which is less than $a$ in length.\\
\indent
When $\phi = \pi$, we will have two circles in the $xz$-plane; one a mirror image of $C$, on the other size of the $z$-axis.

\clearpage

\ulind{Demonstrating that $\sigma$ is a smooth surface:}
First, partial differentiation:
$$
\sigma_\theta = (-b \cos \phi \sin \theta, \ -b \sin \phi \sin \theta, \ b \cos \theta )
$$
$$
\sigma_\phi = (-a \sin \phi - b \sin \phi \cos \theta, \ a \cos \phi + b \cos \phi \cos \theta, \ 0 )
$$

Because $\sigma \in C^\infty$, we can say it is smooth.\\\\
Now, to create an atlas for $\sigma$; its patches will be $\sigma_u$ and $\sigma_{\tilde u}$, and they will both map $U$ and $\widetilde{U}$ to torus $S$, respectively.\\
\indent
Let $U$ and $\widetilde{U}$ be defined as follows:

$$
U = {\theta,\phi \ | \ }
$$




\end{document}
