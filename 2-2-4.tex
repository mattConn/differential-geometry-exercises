\title{Chapter 2, Section 2. Exercises 1, 4-6}
\author{
	MTH 594, Prof. Mikael Vejdemo-Johansson \\
	Differential Geometry Independent Study \\
	\\
	Matthew Connelly \\
}
\date{\today}



\documentclass[12pt]{article}

\usepackage[top=.5in, bottom=.75in, left=1in, right=1in]{geometry}
\usepackage{amssymb}
\usepackage{amsmath}


\begin{document}
\maketitle

\section*{Exercise 2.2.4}
[Part 1]\\
Let $k$ be the signed curvature of a plane curve $C$ expressed in terms of  its arc-length. Show that, if $C_a$ is the image of $C$ under the dilation of $\textbf{v} \rightarrow a\textbf{v}$ of the plane (where $a$ is a non-zero constant), the signed curvature of $C_a$ in terms of \textit{its} arc-length $s$ is $\frac{1}{a}k(\frac{s}{a})$.\\

[Part 2]\\
A heavy chain suspended at its ends hanging loosely takes the form of a plane curve $C$. Show that, if $s$ is the arc-length of $C$ measured from its lowest point, $\phi$ the angle between the tangent of $C$ and the horizontal, and $T$ the tension in the chain, then
$$ Tcos\phi = \lambda \ , \ Tsin\phi = \mu s $$
where $\lambda$, $\mu$ are non-zero constants (we assume that the chain has constant mass per unit length). Show that the signed curvature of $C$ is 
$$ k_s = \frac{1}{a}\left(1+\frac{s^2}{a^2}\right)^{-1}$$
where $a=\lambda/\mu$, and deduce that $C$ can be obtained from the catenary in Example 2.2.4 by applying a dilation and an isometry of the plane.

\vspace{1cm}
\hrule
\vspace{1cm}

[Part 1]\\
Showing that the signed curvature of $C_a$ is $\frac{1}{a}k(\frac{s}{a})$.\\
\\
First, relating the arc-length of $C$ to that of $C_a$:\\
\\
$C = (x, \ y) \ \hspace{1cm}\  C_a = (ax, \ ay)$\\
$k$ is the signed curvature of $C$ in respect to its arc-length.\\
\\
Let $s_c$ be the arc-length of $C$, given $s$ is the arc-length of $C_a$. \\
$\therefore$ \  $s = a \ s_c$ \hspace{1cm}\textrm{{\footnotesize $C_a$'s arc-length $=$ $C$'s arc-length $\cdot$ a}}\\
\\
Proof: \\
\\
$s_c = \int_0^t \parallel C' \parallel dt$ \hspace{2cm}\textrm{{\footnotesize $C$'s arc-length}}\\
$s_c = \int_0^t \sqrt{(x')^2 + (y')^2} \ dt$ \\
\\
\\
$s = \int_0^t \parallel C_a' \parallel dt$ \hspace{2cm}\textrm{{\footnotesize $C_a$'s arc-length}}\\
\\
$s = \int_0^t \sqrt{(ax')^2 + (ay')^2} \ dt$ \\
\\
$s = \int_0^t \sqrt{a^2[\ (x')^2 + (y')^2  \ ]} \ dt$ \\
\\
$s = \int_0^t \sqrt{a^2}\sqrt{ (x')^2 + (y')^2  } \ dt$ \\
\\
$s = \int_0^t a \sqrt{ (x')^2 + (y')^2  } \ dt$ \\
\\
$s = a\int_0^t \sqrt{ (x')^2 + (y')^2  } \ dt$ \hspace{1cm}\textrm{{\footnotesize $a$ is constant; commuted outside of integral expression}}\\
$s = a \ s_c$ \\

Lastly, relating the curvature of $C$ to that of $C_a$:\\
$$k = n_sk_s \hspace{1cm}\textrm{{\footnotesize $C$'s signed curvature = unit normal $\cdot$ tangent's rate of turning}}$$\\
\footnotesize{$C$'s rate of turning $k_s = \dot\phi$ \ :}
\normalsize
$$\phi = tan^{-1}\frac{y'}{x'} \hspace{1cm}\textrm{{\footnotesize Angle between $C$'s tangent and the horizontal}}$$\\
$$\dot\phi = \frac{d\phi}{ds} = \frac{1}{1-(\frac{y'}{x'})^2} \cdot \frac{x'y'' - x''y'}{(x')^2} = k \hspace{1cm}\textrm{\footnotesize{$C$'s tangent's rate of turning}}$$\\
\\
{\footnotesize $C_a$'s rate of turning $\dot\phi_a$ \ :}
$$\phi_a = tan^{-1}\frac{ay'}{ax'} = tan^{-1}\frac{y'}{x'} = \phi \hspace{1cm}\textrm{\footnotesize{Angle between $C_a$'s tangent and the horizontal}}$$\\
$$\therefore \hspace{1cm} \dot\phi_a = \dot\phi = k_s \hspace{1cm}\textrm{\footnotesize{$C$'s rate of turning = $C_a$'s rate of turning}}$$
\\
\\
\normalsize
To relate all of this to $\frac{1}{a}k(\frac{s}{a})$:\\
\\
Let $n_s^a = (-a \ sin\phi, \ a\ cos\phi)$ be $C_a$'s unit normal; a $\frac{\pi}{2}$ rotation of $C_a$'s tangent $\bold{t}$.\\
$n_s$ is $C$'s unit normal.\\
\\
$ k(s_c) = k\left( \frac{s}{a} \right) \hspace{1cm}\textrm{{\footnotesize Because $s$ = $a \cdot s_c$ and $C$ and $C_a$ have the same turning rate}} $\\

$n_s \cdot k_s = n_s^a \cdot k_s \hspace{1cm}\textrm{{\footnotesize Expansion of $k$ at $s_c$ and $\frac{s}{a}$}}$\\

$ (-sin\phi, \ cos\phi) \cdot \dot\phi = (-a \ sin\phi, \ a \ cos\phi) \cdot \dot\phi \hspace{1cm}\textrm{{\footnotesize Further expansion}}$\\

$ (-sin\phi, \ cos\phi) \cdot \dot\phi \cdot \frac{1}{a} = (-sin\phi,  \ cos\phi) \cdot \dot\phi \hspace{1cm}\textrm{{\footnotesize Division by constant of dilation}}$\\

$ k(s_c)\cdot \frac{1}{a} = k\left( \frac{s}{a} \right) \cdot \frac{1}{a}\hspace{1cm}\textrm{{\footnotesize Simplification}} $\\

\vspace{1cm}
\normalsize
[Part 2]\\

Show:\\
$$ Tcos\phi = \lambda \ , \ Tsin\phi = \mu s $$\\
Where $\lambda, \mu$ are non-zero constants.\\
\\
Curve $C$ is described as a heavy chain with tension $T$; a catenary.\\
$\therefore  \ C = (t, \ cosht), \ C' = (1, \ sinht)$\\
\\
Unit tangent vector: $\bold{t} = (cos\phi, \ sin\phi)$
\\
Arc-length $s = sinht$, because $s = \int_0^t\sqrt{1+cosh^2t} = sinht$  \ for a catenary.\\
$$C' = (1, \ sinht)$$
$$C' = \bold{t} \hspace{1cm}\textrm{{\footnotesize Relating $C'$ to arc-length of $C$}}$$
$$TC' = T\bold{t} \hspace{1cm}\textrm{{\footnotesize Scaling by constant of tension}}$$
$$TC' = (Tcos\phi, \ Tsin\phi) \hspace{1cm}\textrm{{\footnotesize Expansion of $T\bold{t}$}}$$
$$\therefore \ TC' = (\lambda, \ \mu s) \hspace{1cm}\textrm{{\footnotesize Expansion of $T\bold{t}$}}$$
\\
\\
Also show:
$$ k_s = \frac{1}{a}\left(1+\frac{s^2}{a^2}\right)^{-1}$$
The signed curvature of $C$ is $k_s$.
\\
\\
Given: $a = \frac{\lambda}{\mu}$\\
From earlier definitions of $\lambda$, $\mu$:\\
$$\mu = \frac{Tsin\phi}{s}$$
$$ \therefore \ a = \frac{Tcos\phi}{Tsin\phi} \cdot s $$
Which can be simplified:
$$ a = \frac{cos\phi}{sin\phi} \cdot s \hspace{1cm}\textrm{{\footnotesize Cancellation of $T$}}$$
$$ a = \frac{1}{tan\phi} \cdot s $$
$$ a = \frac{1}{tan\phi} \cdot sinh \hspace{1cm}\textrm{{\footnotesize Expansion of arc-length $s$}}$$
$$ \therefore \ a = \frac{sinh}{tan\phi} $$
\\
However,\\
$ tan\phi = sinh $, making $a = \frac{\lambda}{\mu} = 1$.\\

Justification:\\
$tan\phi = \frac{y'}{x'} \hspace{1cm}\textrm{{\footnotesize tan of angle between $\bold{t}$ and $x$-axis }} $\\
$tan\phi = sinh \hspace{1cm}\textrm{{\footnotesize}} $\\

Now expanding $a$ and $s$ in $k_s$:
$$ k_s = \frac{1}{\left(1+sinh^2t\right)}$$

\end{document}
This is never printed