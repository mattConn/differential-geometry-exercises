\title{Chapter 4, Section 2. Exercises 1, 2, and 4 through 9}
\author{
	MTH 594, Prof. Mikael Vejdemo-Johansson \\
	Differential Geometry Independent Study \\
	\\
	Matthew Connelly \\
}
\date{\today}



\documentclass[12pt]{article}

\usepackage[top=.5in, bottom=.75in, left=1in, right=1in]{geometry}
\usepackage{amssymb}
\usepackage{amsmath}
\usepackage{graphicx}
\usepackage{subcaption}

\newcommand{\ulind}[1]
{
\noindent
\underline{#1}\\\\
\indent
}


\begin{document}
\maketitle

\section*{Exercise 4.2.8}
\indent

Show that translations and invertible linear transformations of $\mathbb{R}^3$ take smooth surfaces to smooth surfaces.

\vspace{1cm}
\hrule
\vspace{1cm}
\noindent

Translations and invertible linear transformations induce diffeomorphisms on surfaces; a diffeomorphism being an isomorphism (an invertible morphism or mapping that preserves length) between two smooth manifolds. \\

Let $S$ be a surface, and $\widetilde{S}$ be $S$ subject to a translation or invertible linear transformation.\\\\
\indent
Because $S$ is a smooth bijective mapping of $U$ to $S \cap W$, $\widetilde{S}$ will be smooth as well.\\\\
\indent
Because these two surfaces are smooth and a diffeomorphism is between two smooth manifolds (and we are trusting that $S$ is locally Euclidean), we can then see that a translation/invertible linear transformation of a surface will preserve that surface's smoothness.


\end{document}
