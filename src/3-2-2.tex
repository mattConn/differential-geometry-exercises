\title{Chapter 3, Section 2. Exercises 1 and 2}
\author{
	MTH 594, Prof. Mikael Vejdemo-Johansson \\
	Differential Geometry Independent Study \\
	\\
	Matthew Connelly \\
}
\date{\today}



\documentclass[12pt]{article}

\usepackage[top=.5in, bottom=.75in, left=1in, right=1in]{geometry}
\usepackage{amssymb}
\usepackage{amsmath}
\usepackage{graphicx}
\usepackage{subcaption}


\begin{document}
\maketitle

\section*{Exercise 3.2.2}

By applying the isoperimetric inequality to the ellipse
$$
\frac{x^2}{p^2}+\frac{y^2}{q^2} = 1
$$
(where $p$ and $q$ are positive constants), prove that
$$
\int_{0}^{2\pi}\sqrt{ \, p^2 \ sin^2t + q^2 \ cos^2t \ } \ dt \ \geq 2\pi \sqrt{pq} \ ,
$$
with equality holding if and only if $p = q$.

\vspace{1cm}
\hrule
\vspace{1cm}

\noindent
Let $\gamma(t) = (p \ cost, q \ sint)$ be our ellipse.\\\\
Then, differentiate $\gamma$ and find its norm:
$$
\gamma'(t) = (-p \ cost, q \ cost)
$$
$$
||\gamma'(t)|| = \sqrt{p^2 \ sin^2t + q^2 \ cos^2t}
$$
\\
We know the following definitions of $l(\gamma)$ and $A(\gamma)$:
$$
A(\gamma) = \frac{1}{2}\int_{0}^{T} \left( xy' - yx' \right) dt
$$
$$
l(\gamma) = \int_{0}^{T}||\gamma'(t)||  \ dt
$$
$\gamma$ is $T$-periodic.

\clearpage
We can expand these definitions using our current definition of $\gamma$:
$$
A(\gamma) = \frac{1}{2}\int_{0}^{2\pi} \left( pq \ cos^2t + pq \ sin^2t \right) dt
$$
$$
l(\gamma) = \int_{0}^{2\pi}\sqrt{p^2 \ sin^2t + q^2 \ cos^2t}  \ \ dt
$$
Simplifying $A(\gamma)$:
$$
A(\gamma) = \frac{1}{2}\int_{0}^{2\pi} pq \ dt = \pi \ pq
$$
\\
Then we can use the isoperimetric inequality:
$$
A(\gamma) \leq \frac{l(\gamma)^2}{4\pi}
$$
After expansion:
$$
\pi \ pq  \ \leq \ \frac{ \left( \int_{0}^{2\pi}\sqrt{p^2 \ sin^2t + q^2 \ cos^2t} \ dt \right)^2 }{4\pi}
$$
$$
4\pi^2 \ pq  \ \leq \  \left( \int_{0}^{2\pi}\sqrt{p^2 \ sin^2t + q^2 \ cos^2t} \ dt \right)^2
$$
$$
2\pi \ \sqrt{pq}  \ \leq \ \int_{0}^{2\pi}\sqrt{p^2 \ sin^2t + q^2 \ cos^2t} \ dt
$$
\\
\noindent
\underline{When $p=q$:}\\
\indent

Let radius $R = p = q$.

$$
2\pi \ \sqrt{R^2}  \ \leq \ \int_{0}^{2\pi}\sqrt{R^2} \ dt
$$
$$
2\pi \ R  \ \leq \ R \ \int_{0}^{2\pi}1 \ dt \hspace{1cm}\textrm{{\footnotesize $R$ is constant.}}
$$
$$
2\pi \ R  \ = \ R \ 2\pi
$$
\end{document}
This is never printed