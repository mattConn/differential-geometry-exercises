\title{Chapter 4, Section 2. Exercises 1, 2, and 4 through 9}
\author{
	MTH 594, Prof. Mikael Vejdemo-Johansson \\
	Differential Geometry Independent Study \\
	\\
	Matthew Connelly \\
}
\date{\today}



\documentclass[12pt]{article}

\usepackage[top=.5in, bottom=.75in, left=1in, right=1in]{geometry}
\usepackage{amssymb}
\usepackage{amsmath}
\usepackage{graphicx}
\usepackage{subcaption}

\newcommand{\ulind}[1]
{
\noindent
\underline{#1}\\\\
\indent
}

\newcommand{\R}
{
\mathbb{R}
}

\begin{document}
\maketitle

\section*{Exercise 4.2.9}
\indent

Show that every open subset of a smooth surface is a smooth surface.

\vspace{1cm}
\hrule
\vspace{1cm}
\noindent

A surface $S \in \R^3$ is defined in the context of an open $\R^2$ set $U$ and an open $\R^3$ set $W$ where $S \cap W$ is homeomorphic to $U$; this homeomorphism can be written as the following smooth bijective mapping:

$$
\sigma:U \rightarrow S \cap W
$$

$S$ will be an atlas consisting of these homeomorphisms which are the patches of the atlas.\\
\indent
$\therefore$, taking an open subset of $S$ consisting of one or more patch (a partial atlas) will produce another surface.\\

\indent
If $S$ is smooth, then it is continuously differentiable all over; there should be no subset of $S$ then that is \emph{not} smooth.\\
\indent
Lastly, the patches of a surface are defined as being open; the subset of a surface will then be a surface itself, provided it is open, much in the way a patch must be open.

\end{document}
